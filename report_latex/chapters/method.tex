%----------------------------------------------------------------------------------------
% Method
%----------------------------------------------------------------------------------------
\section{Method}\label{sec:method}

\subsection{Hardware}

The Energy Gateway system consists of the following hardware components:

\begin{itemize}
  \item ESP32 devkit board
  \item Heltec Wireless Stick
  \item Breadboard
  \item Connecting wires
\end{itemize}

The ESP32 devkit board is a low-cost microcontroller board that is based on the ESP32 microcontroller. It was chosen due to its low cost, integrated Wi-Fi and Bluetooth capabilities~\cite{platformio:esp32-devkitv1}.

Any device could have been used to simulate a serial device, but the Heltec Wireless Stick was chosen due to availability and ease of use.

\subsection{WiFi Provisioning through Bluetooth}

The WiFi provisioning through Bluetooth process involves the ESP32 module starting a BLE server with a predefined service UUID and advertising itself as a WiFi provisioning device. A BLE-enabled device connects to the ESP32 module using its service UUID, and communication between the devices is established. The BLE-enabled device requests a list of available WiFi networks using the \texttt{wifi\_scan} endpoint, and the ESP32 module responds with a list of SSIDs and RSSIs using the same endpoint. The BLE-enabled device selects a WiFi network and sends its SSID and password to the ESP32 module using the \texttt{wifi\_config} endpoint. The ESP32 module validates the credentials, attempts to connect to the selected WiFi network, and sends a connection status message to the BLE-enabled device using the same endpoint \cite{espressif:esp-idf-programming-guide}.

The WiFi provisioning through Bluetooth method requires applications on both the ESP32 module and the BLE-enabled device, and the Google Play store offers the Espressif Android app that serves as a client \cite{google-play:esp-ble-provisioning}.

\subsection{OTA Updates}

The OTA update system was developed to allow for remote firmware updates of the ESP32 board via a Wi-Fi connection. To implement this functionality, the partition table of the device was configured with at least two "OTA app slot" partitions and an "OTA Data Partition". An esp-idf component, \texttt{energy\_gateway\_ota}, was created to handle the initialization, download, verification, and activation of new firmware images from a given URL. This component was integrated into the \texttt{main.c} file to periodically check for updates and perform them if available. The firmware image was hosted on a web server configured to serve the firmware image over HTTPS (refer to Figure \ref{fig:web_server}). The \texttt{energy\_gateway\_ota} component downloaded and verified the integrity of the firmware image and activated it by setting the boot partition to the OTA app slot that was not previously selected for booting. The system also enables the possibility for rollback to a previous firmware image in the event of a failed update.

\subsection{UART communication}

The UART protocol was utilized to enable serial communication between the Energy Gateway system and a Heltec Wireless Stick (refer to Figure \ref{fig:uart_connection}). The ESP32 microcontroller was programmed to request data from the Heltec Wireless Stick at a rate of 20Hz, while the Heltec Wireless Stick transmitted a random number between 0 and 255 to the ESP32 module every 50ms. The ESP32 module then evaluated whether the received data was too high or low, sending back a 1 or 0, respectively, to the Heltec Wireless Stick.

\subsection{Concurrency}

In our project, we created two tasks, the UART task and the OTA task. The UART task reads data from a Heltec Wireless Stick, sends data back to it, and is given a priority level of \texttt{ESP\_TASK\_PRIO\_MAX - 6} (level 19), which is a highly prioritized level (refer to Table \ref{table:task-priorities}). The priority level ensures that the UART task receives immediate attention and is not preempted by other lower-priority tasks. The OTA task checks for new firmware images, updates the firmware image if a new one is available, and is given a priority level of 2, which is a low-priority level. This priority level ensures that the OTA task does not preempt critical tasks.

In addition to the two tasks, we implemented a timer to ensure that the OTA task runs at a set interval. The timer is set to run the OTA task once every 24 hours, ensuring that the firmware is always up to date while avoiding excessive updates. When the timer expires, the OTA task is once again placed in the task list, with a priority level of 2, to check for new firmware images and update the firmware image if a new one is available while avoiding the interruption of other more ciritical tasks.