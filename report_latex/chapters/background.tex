%----------------------------------------------------------------------------------------
% Background
%----------------------------------------------------------------------------------------
\section{Background}
\label{sec:background}

\subsection{Technical background}
\label{sec:technical}
ESP32 is a low-cost, low-power system on a chip (SoC) microcontroller that is widely used in Internet of Things (IoT) applications due to its built-in Wi-Fi and Bluetooth capabilities \cite{espressif:popularity} \cite{espressif:esp32_datasheet}.

\subsubsection{Wi-Fi Provisioning}
\label{subsec:bluetooth}
To make the initial setup of the ESP32 as easy as possible, the project aims to use Bluetooth and a mobile device to enable easy provisioning of the device to a Wi-Fi network.

The Wi-Fi provisioning via Bluetooth will allow the end user to input the Wi-Fi SSID and password, without the need for a computer.

\subsubsection{OTA Updates}
\label{subsec:ota}
OTA updates allow firmware or software updates to be installed on an ESP32 device over a network connection, without the need for a physical connection to a computer. However, OTA updates can be challenging to implement and can result in unexpected interruptions and errors if not handled correctly \cite{Arakadakis:2021}.

The project will focus on developing a reliable OTA update system that can be integrated into the ESP-IDF framework.

\subsubsection{UART Communication}\label{subsec:uart}
The Energy Gateway's final product requires the capability to poll from and send data to serial devices. To enable serial communication, UART (Universal Asynchronous Receiver/Transmitter) can be used, which enables asynchronous communication between two devices in a full-duplex manner~\cite{philips:uart}. The ESP32 has three UART interfaces, UART0, UART1 and UART2, which can be used to communicate with the serial device~\cite{espressif:esp32_datasheet}.

\subsubsection{Concurrency}
\label{subsec:concurrency}
Efficient task management is crucial when developing software using the Espressif IoT Development Framework (ESP-IDF). The underlying operating system for ESP-IDF is FreeRTOS, which is a real-time operating system designed for microcontrollers that provides a multitasking environment for running multiple tasks simultaneously on a single processor. One essential feature of FreeRTOS is the priority-based preemptive scheduling algorithm that allows for efficient task management. ESP-IDF utilizes a range of 26 priority levels, numbered from 0 (lowest) to 25 (highest), which developers can use to optimize system performance and ensure that critical tasks receive the necessary resources to execute efficiently (refer to Table \ref{table:task-priorities}). Understanding the criticality and potential impact of a task on the system is crucial in selecting an appropriate priority level~\cite{Davis:2016}.

\subsection{Social background}
\label{sec:social}
The increasing popularity of IoT devices has led to a growing demand for systems that can be easily and reliably updated over the air \cite{Villegas:2019}. This is particularly important for devices that are deployed in remote or hard-to-reach locations, where physical access for updates may not be possible or practical. By developing a system for reliable OTA updates, this project aims to address this need and make it easier for end-users to maintain and update their ESP32 devices.

\subsection{Previous research}
\label{sec:previous}
The project will build on previous work into OTA updates and Wi-Fi provisioning for ESP32 devices \cite{espressif:esp-idf-programming-guide}. While previous work has focused on individual components ot these processes, the present project combines these features using ESP-IDF components with the additional capability of reading from an external device using UART communication.