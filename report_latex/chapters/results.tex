%----------------------------------------------------------------------------------------
% Results
%----------------------------------------------------------------------------------------
\section{Results}\label{sec:results}

\subsection{WiFi Provisioning through Bluetooth}

The WiFi provisioning through Bluetooth system was developed and implemented on the ESP32 microcontroller. The system allows for efficient and reliable configuration of WiFi credentials without requiring code modification. The use of a BLE-enabled device streamlines the configuration process, and the system was tested and proven to be effective. The ESP-IDF framework was utilized to integrate the system as a component. Overall, this method offers a potential solution in the realm of IoT, providing secure and efficient communication between devices.

\subsection{OTA Updates}

The developed OTA update system successfully enables remote firmware updates of the ESP32 board via a Wi-Fi connection. The system utilizes the \texttt{energy\_gateway\_ota} component to handle the initialization, download, verification, and activation of new firmware images from a given URL. The system was tested, and it was found to be reliable and effective in updating the firmware image without interrupting any critical tasks that the unit may be currently running. The system also integrates git tags as a versioning system, resulting in the update being aborted if the tags match. In addition, the system enables the possibility for rollback to a previous firmware image in the event of a failed update. However, this rollback functionality was not implemented in our project.

\subsection{UART Communication}

The ESP32 microcontroller was successfully connected to the Heltec Wireless Stick and was able to establish a serial communication channel between the two devices (refer to Figure \ref{fig:serial_connection}). The Wireless Stick was programmed to send random integers between 0 and 255 over serial at a fixed interval, while the ESP32 microcontroller utilized the \texttt{uart\_read\_bytes()} function to read the incoming bytes with a timeout of 20 milliseconds (refer to Figure \ref{fig:uart_read_bytes_function}). After reading the bytes, the microcontroller evaluated the received data, sending back a 1 or 0 to the Wireless Stick based on the data being too high or low, respectively.

The sending interval was tested at different frequencies, with promising results. The device was able to catch all values sent at a data rate of 33Hz when no other tasks were running (refer to Figure \ref{fig:serial_communication_33hz}). When the OTA task was enabled and ran every five seconds, the device was able to catch all values sent when the data rate was reduced to 20Hz (refer to Figure \ref{fig:serial_communication_20hz}). The UART communication feature was successfully implemented and tested under stress conditions, ensuring that the feature functions reliably in the final Energy Gateway product.

\subsection{Concurrency}

The project successfully implemented task management with ESP-IDF utilizing priority levels and timers. Priority levels where carefully selected for each task based on its criticality and potential impact on the system. The UART task was given a higher priority level to ensure that it receives immediate attention, while the OTA task was given a lower priority level to avoid preempting critical tasks. By implementing a timer, we ensured that the OTA task runs at regular intervals, keeping the firmware up to date while avoiding excessive updates. We have shown that selecting an appropriate priority level for each task and utilizing timers is crucial in optimizing system performance and ensuring efficient task management.