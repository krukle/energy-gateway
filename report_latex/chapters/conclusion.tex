%----------------------------------------------------------------------------------------
% Conclusion
%----------------------------------------------------------------------------------------
\section{Conclusion}
\label{sec:conclusion}

In conclusion, our project has successfully implemented and tested key features for the Energy Gateway system, which is a microcontroller-based system designed to help stabilize the power grid. Our work has shown that the implemented features have the potential to enhance the usability and security of the system and enable easy deployment of updates, which is critical in the realm of IoT. We have also demonstrated our ability to program with the ESP-IDF framework and utilize the FreeRTOS real-time operating system effectively.

However, there are limitations and possible sources of error that should be taken into consideration, and further testing and improvements could enhance the reliability and performance of these features. Specifically, future research could investigate the implementation of a more robust versioning system that can handle more complex scenarios, as well as the potential to implement a rollback mechanism for the OTA update feature to ensure that the device is always running a stable version of the firmware. Additionally, the use of a more fine-grained update mechanism, where each component update only overrides its own component binary, could reduce the risk of the device being left in a state where it is unable to run any of its components. Overall, these areas of future research could expand the knowledge base in this area and contribute to the development of more reliable and efficient microcontroller-based systems for energy management and control.